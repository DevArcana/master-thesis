\abstract{
    % English abstract 

    The game world is a scene on which events play out.
    It is usually composed of a plethora of regions, each having distinct characteristics and associated lore.
    When a player performs any action that has influence on the state of the game, it is expected for the news of it to reach other parts of the game world and be referenced in dialogues or perhaps even have visible impact across the land.
    The scope of the influence depends on the scale of the event.
    A dragon flying past a mountain should be the main topic of discussion in the entire region while an apple being stolen from a local grocer's should not be referenced outside the village the act of thievery took place in.
    This thesis describes the ways to model information about events in the game world and the way it should propagate across the map.

}{
    % Abstract translated into Polish

    Świat gry to scena, na której rozgrywają się wydarzenia.
    Zazwyczaj składa się z wielu regionów, z których każdy posiada odrębne cechy i związane z nimi lore.
    Kiedy gracz wykonuje jakąś akcję, która ma wpływ na stan gry, oczekuje się, że wieści o niej dotrą do innych części świata gry i zostaną przywołane w dialogach, a może nawet będą miały widoczny wpływ na całą krainę.
    Zakres wpływu zależy od skali wydarzenia.
    Smok przelatujący obok góry powinien być głównym tematem dyskusji w całym regionie, podczas gdy jabłko skradzione z lokalnego sklepu spożywczego nie powinno być przywoływane poza wioską, w której akt złodziejstwa miał miejsce.
    Niniejsza rozprawa opisuje sposoby modelowania informacji o wydarzeniach w świecie gry oraz sposób, w jaki powinna ona rozchodzić się po mapie.

}