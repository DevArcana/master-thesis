\Chapter{Related Work}\label{chapter:related}

All events happening within the game world form the plot of the game.
The way the plot is presented to the player depends on the narrative.
In other words the narrative is the plot with added context and presented using a certain storytelling technique.
The reception of the narrative by the player is often less dependent on the plot than it is on the technique used to present it.
In this thesis, the focus is not on the narrative but rather just the formal representation of the plot itself.
The plot consists of all events, characters and their actions in the game, all of which do not need to be available to the player or even exist in any way as separate beings.
Characters and their actions can be parts of the plot in the form of writing or recordings and events can be referenced in the history of the simulated world even though they were never really represented.
The actual form of the plot depends on the type of the game and the narrative it tries to establish.
As noted by Lindley, the plot state of the game world does not need to be available and presented to the player in its entirety, as is the case for example in games such as SIM City \cite{lindley2005story}.
In many games the player experiences a subset of the available plot and thus forms their own customized narrative.

Saussure\cite{gordon2004langue} introduced the distinction between language (fr. \emph{la langue}) and meaning (fr. \emph{la parole}).
This distinction is similar to the difference between plot and narrative and likewise applicable in the context of plot representation alone.
In Saussure's framework, language refers to the overall system of rules, conventions, and structures that govern a particular language, while meaning refers to the individual utterances or expressions produced by speakers of that language.
This distinction emphasizes the idea that language is a system that is shared and agreed upon by a community of speakers, and that meaning is not fixed but rather is created through the interaction between the language system and the speaker's intentions and context.
When representing the plot state of a game world, it is important to consider the language used to describe that world and the range of expressions that are possible within that language.

A plot in a game can be implemented using a script by the game designer or it can emerge naturally during gameplayer based on the interactions between individual elements in game.
The former approach is usually implemented in role playing games as game designers and writers collaborate to bring to life the creative vision for a given imaginary world.
On the other hand, games such as Dwarf Fortress use simulation of individual interactions between game elements to formulate a plot \cite{adams2015simulation}.
This in turn leads to a phenomenon called emergent behavior where an interaction of simple elements produces a complex behaviour \cite{adams2019emergent}.
Emergence in game design is not a new concept and is used frequently in the game development industry \cite{sweetser2008emergence}.
While it can increase immersiveness and replayability, it also increases the overall complexity of both gameplay and development.

Games that lean heavily towards utilization of emergent behaviours usually depend on multi-agent simulation where each agent represents either a non-playable character (NPC) or the player themselves.
Agents that are controlled by the game and not the player have builtin interaction rules and behaviours that allow them to act in a way that is expected and realistic in a given scenario.
Agent-based simulation (ABS) has been used for a long time already and can be applied to any given game genre including some well known zero-player games such as Game of Life \cite{chan2010simulation}.
In these simulations agents have their own mememory and ability to process information, eventaully integrating it into knowledge that is used for choosing the response to any situation that can happen within the rules of the game.

% What is data, information, knowledge?

To understand how agents work with information, the concept of information and its relationship to knowledge needs to be explained.
The earliest work discussing information and similar concepts comes from Shannon in 1948\cite{shannon1948mathematical}.
The author however never defines the term itself and simply proposes models for quantitative analysis.
Approach described back then was more related to data than it was to information.
The distinction was made clear by Langefors in 1973\cite{langefors1973theoretical}.
Langefors proposed an approach putting the users of information first and proposed the following equation to describe the interaction of time, data, knowledge and information:
$$
    I = i(D, S,t)
$$
where:
\begin{variables}
    $I$ & information \\
    $i$ & interpretation process \\
    $D$ & data \\
    $S$ & knowledge \\
    $t$ & time
\end{variables}
In 1989, Ackoff\cite{ackoff1989data} defined the hierarchy of the human mind as consisting of wisdom at the very top and descending into understanding, knowledge, information and further into data.
The more popularized variant of this hierarchy being DIKW (data, information, knowledge, wisdom), became widespread in literature regarding knowledge management\cite{skyrme2007knowledge}.
There have been many attempts to define the distinction between these terms as well as the terms themselves.
One notable example was done by Grabowski and Zając in 2009\cite{mariusz2009dane}.
\begin{figure}[h]
    \centering
    \includegraphics[width=0.6\textwidth]{images/od_nadawcy_do_odbiorcy.png}
    \caption{Information flow\cite{mariusz2009dane}}\label{fig:od_nadawcy_do_odbiorcy}
\end{figure}
The authors stress the hardship of definition of such primal terms and highlight the two major science disciplines that are related to them (information theory and knowledge management).
Both disciplines use the pyramid of knowledge (or information respectively).
According to Liew\cite{liew2013dikiw} trying to accurately define data, information and knowledge will result in circular definition.
To solve that problem he proposes a DIKIW model where "I" stands for "Intelligence".
Irrespective of chosen framework, everyone agrees that information and knowledge are related but represent different levels of abstraction.
While data is raw and unprocessed before it can become information, similarly information needs to be retained and assimilated to become knowledge.
The scope of this thesis covers only information understood as statements of fact and knowledge defined as retained and processed information.

A population of people can be said to possess knowledge similarly to a single individual.
Often a crowd has greater knowledge than any individual.
Surowiecki said "under the right circumstances, groups are remarkably intelligent, and are often smarter than the smartest people among them"\cite{surowiecki2005wisdom}.
To understand how knowledge is gained within a population one should analyze the way information itself is propagated.
In fact, people have already noticed that certain facts, stories or news share similarities with viral infections in how they spread across the whole population and eventually die out.
One such example is the work done by Liu et al.\cite{liu2016} where the authors describe how information spread can be explained using epidemiological models.
Truly, the simplest of such models called SIR\cite{weiss2013sir} can be used to describe in simple terms how a population can be partitioned into groups of \emph{Susceptible} (people who haven't encountered the viral information yet), \emph{Infected} (those who know the news and are willing to share it) and \emph{Removed} (everyone who either lost interest or simply never exhibited it).
There are however differences between infectious diseases and information.
For this reason variations of the SIR model became born.
Zhao et al.\cite{zhao2012sihr} have developed the SIHR model where H stands for \emph{Hibernators}.
The authors stress the importance of forgetting and remembering in trying to model the spread of information.
Another variation worth noting is the SCIR model consisting of \emph{Susceptible}, \emph{Contacted}, \emph{Infected} and \emph{Refractory}\cite{xiong2012scir}.
Both models use complex systems in order to simulate information propagation.
Other attempts at modelling information propagation are networks\cite{rodriguez2013} and cellular automata\cite{silva2020}.
Most literature dealing with the topic refers to social media, blogs and internet as the main medium of information exchange.
Relatively few works analyze the mechanisms behind word-of-mouth communication.
In 2003 at the "Symposium on Applications and the Internet", Takeuchi, S. and Kamahara, J. and Shimojo, S. and Miyahara, H. have presented their work titled "Human-network-based filtering: the information propagation model based on word-of-mouth communication" which deals with simulating how certain information is spread based on the filters of interest\cite{takeuchi1183031}.
Most authors treat knowledge of information as a binary value.
In reality the level of knowledge is often more complicated.
Silva et al.\cite{silva2020} propose the application of cellular automata to model information expressed as a linear value.

When trying to model how information propagates in a given system, it is often helpful to try and list its characteristics.
Grabowski and Zając followed the steps of Langefors and others while describing the following set of information characteristics:
\begin{itemize}
    \item Completedness - information must inlude context necessary for its interpretation
    \item Versatility - information should be interpretable from multiple viewpoints
    \item Accuracy - it should not be too vague nor too specific
    \item Financially viable - it should be useful in business context
\end{itemize}
These show clearly how most work done previously in this domain of knowledge related to the business context.

It is apparent that information as utilized in agent-based simulation games is different than the concept used in information theory and derived works.
Social agents that interact with each other and simulate conversations, emotional state and organic behaviour have been analyzed since well before the development of artificial intelligence and language models.
A notable example is the work done by Grey et al. in 2011 titled "Procedural quests: A focus for agent interaction in role-playing-games"\cite{grey2011procedural}.
The author describes the problems with contemporary games and their shallow plot propagation mechanisms and proposes their approach of modelling plot and its propagation using a social multi-agent system.
Very recently many articles have been published that utilize the developments in the area of language models and natural language processing in general.
Park et al.\cite{park2023generative} have used the GPT-4 model\cite{openai2023gpt4} to simulate a small village populated by characters each with individual personality and motivations.
The characters held conversations, organized meetings and socialized in their free time.

In conclusion it can be seen that the use of agent-based simulation in game development is a growing field of research.
Artificial intelligence and its derivatives drive the current state of the art in this area.
Alternative approaches such as proposed back in 2011 by Grey\cite{grey2011procedural} should be explored to compete with the indeterministic nature of language models or perhaps to become the benchmarks against them.