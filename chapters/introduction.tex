\Chapter{Introduction}\label{chapter:introduction}
People experience reality through the lens of their accumulated experience.
Each event in the life of a person adds another perspective on any following event.
Our behaviours are dictated by the way we perceive reality.
That perception is a combination of external sensory information and internal state of mind.
Our emotions, values and beliefs can fundamentally change the interpretation of any given situation.
As we grow old we look back on the choices we made and judge them by the standards of the present.
This simple fact means that there is no objectivity in how humans perceive reality.
Some situations or statements are simply more common in interpretation that others and thus we call them facts.
Others we refer to as opinions and some can be called delusions.
In games, the player is often tasked with accomplishing a goal which puts the future of the world at stake.
The choices made by the player should feel impactful to the overall narrative or the player will feel disengaged from the story, breaking immersion.
Often, choices can be morally ambiguous and the player should face the consequences of their actions in one way or another.
In society crimes are punishable on the basis of proof.
Proof is knowledge backed up by evidence, ergo a type of transferrable information.
An eyewitness of a crime can be untrustworthy and thus a crime might not be punished.
In games this mechanism is often simplified and a crime is detected and propagated across all the land if the player is briefly witnessed even by a single person.
The same goes for other player choices and game events.
People start referencing in-game events in dialogues instantly after they happen as if seemingly everyone became aware of them instantenously.
This can break immersion and player expectations.

\section{Problem Statement}
Propagation of knowledge or in general information is often too simplified in games.
Player actions should generate information that can be perceived, interpreted and propagated naturally by the other agents in the simulated game world.
Interpretation of received (objective) information produces (subjective) knowledge.
The distinction allows for effects such as misinterpretation and thus propagation of false information in the system.
This can lead to interesting mechanics and interactions that simulate reality more closly, increasing player engagement and immersion.
The main topic of research of this thesis is analysis of the three aspects of information in a game world, namely:

\begin{description}
    \item[perception] how, what and when information is received
    \item[interpretation] how information becomes knowledge
    \item[propagation] how knowledge becomes information and propagates to other people
\end{description}

The analysis as well as simulations and other research takes into account the overall end goal of the thesis being implementation in games where computing power is limited and real-time performance is paramount.
In a world with access to information in real-time across the globe via the use of internet, all models will be inherently different in comparison to a world without any such technology.
This thesis takes into account medieval style worlds with fantasy elements such as magic or mythical creatures but no advanced communication techniques.

\section{Thesis Objectives}
To properly understand the three aspects of information previously mentioned, the distinction between information and knowledge needs to be made.
Information itself needs to be defined as well.
To model the behaviour of different models, a cellular automata approach is chosen.

\section{Thesis Outline}
Zarysuj strukturę swojej pracy dyplomowej. Ogólnie przedstawienie pracy. Przykładowo: ,,Praca dzieli się na $7$ rozdziałów (\dots)''. Rozdział \ref{chapter:politechnika} dotyczy (\dots). Temat został rozwinięty~w~\ref{chapter:podrozdzial}.