\Chapter{Introduction}\label{chapter:introduction}
% Start with stories
Since ancient times humans loved stories.
Some stories describe historical events, others are works of fiction.
Some inspire and teach, others let one escape from hardships of day to day life.
All stories have one thing in common.
One thing differentiates them from other written forms.
That thing is the existence of a plot.
% Define plot
Plot is a collection of events, usually in chronological order.
A plot can be realistic, based in history or it can be entirely made up.
It can be said that it is the plot that makes the story.
It contains all the characters, their actions and thoughts as well as events within the story.
% How were stories told
Before people invented the first writing system stories were being told mouth-to-mouth.
Oral traditions continued even after the era of the written word.
Most people have heard the tales of Odysseus and his epic journey.
It is commonly attributed to the ancient poet Homer.
His work inspired both ancient and contemporary historians\cite{marincola2007odysseus}.
% Books and language
After writing got popular among common folk, the spread of storytelling in its written form took off.
In the current day, anyone can write and share the fruits of one's creative imagination with the rest of the world.
Even languages are becoming less of a barrier as not only a large portion of the population speaks English but translation technology and globalization make intercultural and international communication easier than ever\cite{coulmas1987speak}.
But writing itself is not the only way to tell a story.
% Theater
Going back to ancient Greeks, one can find the wealth of plays written to be performed on stage in the first theaters or, more precisely, amfitheaters.
Actors on stage wore masks that obscured their real faces and instead displayed a chosen emotion.
This allowed the people watching from the far seats to still understand the scene.
The aim of these performances was called katharsis\cite{hart2010art}.
Soon, the theater as is known today became born and people have been writing plays for a multidute of reasons.
% Painting
% Radio
% Cinema
% Television
% Interactive media
Books, plays, television and radio are all passive forms of media where content can be consumed by everyone in the same way.
People
% lepsze utożsamianie się z bohaterem, możliwość kszatłtowania historii, w momencie gdy mamy wpływ, stajemy się bohaterem
% reprezentacja grup społecznych
% dopamina - osiągnięcia bohaterów stają się naszymi
% potrzeba tworzenia sztuki i opowiadania historii, na piramidzie potrzeb jest potrzeba samorealizacji, w tym tworzenia
% Games
% Information
People experience reality through the lens of their accumulated experience.
Each event in the life of a person adds another perspective on any following event.
Our behaviours are dictated by the way we perceive reality.
That perception is a combination of external sensory information and internal state of mind.
Our emotions, values and beliefs can fundamentally change the interpretation of any given situation.
As we grow old we look back on the choices we made and judge them by the standards of the present.
This simple fact means that there is no objectivity in how humans perceive reality.
Some situations or statements are simply more common in interpretation that others and thus we call them facts.
Others we refer to as opinions and some can be called delusions.
In games, the player is often tasked with accomplishing a goal which puts the future of the world at stake.
The choices made by the player should feel impactful to the overall narrative or the player will feel disengaged from the story, breaking immersion.
Often, choices can be morally ambiguous and the player should face the consequences of their actions in one way or another.
In society crimes are punishable on the basis of proof.
Proof is knowledge backed up by evidence, ergo a type of transferrable information.
An eyewitness of a crime can be untrustworthy and thus a crime might not be punished.
In games this mechanism is often simplified and a crime is detected and propagated across all the land if the player is briefly witnessed even by a single person.
The same goes for other player choices and game events.
People start referencing in-game events in dialogues instantly after they happen as if seemingly everyone became aware of them instantenously.
This can break immersion and player expectations.

\section{Problem Statement}
Propagation of knowledge or in general information is often too simplified in games.
Player actions should generate information that can be perceived, interpreted and propagated naturally by the other agents in the simulated game world.
Interpretation of received (objective) information produces (subjective) knowledge.
The distinction allows for effects such as misinterpretation and thus propagation of false information in the system.
This can lead to interesting mechanics and interactions that simulate reality more closly, increasing player engagement and immersion.
The main topic of research of this thesis is analysis of the three aspects of information in a game world, namely:
% Add also representation of knowledge
\begin{description}
    \item[perception] how, what and when information is received
    \item[interpretation] how information becomes knowledge
    \item[propagation] how knowledge becomes information and propagates to other people
\end{description}

The analysis as well as simulations and other research takes into account the overall end goal of the thesis being implementation in games where computing power is limited and real-time performance is paramount.
In a world with access to information in real-time across the globe via the use of internet, all models will be inherently different in comparison to a world without any such technology.
This thesis takes into account medieval style worlds with fantasy elements such as magic or mythical creatures but no advanced communication techniques.

\section{Thesis Objectives}
To properly understand the three aspects of information previously mentioned, the distinction between information and knowledge needs to be made.
Information itself needs to be defined as well.
To model the behaviour of different models, a cellular automata approach is chosen.

\section{Thesis Outline}
Zarysuj strukturę swojej pracy dyplomowej. Ogólnie przedstawienie pracy. Przykładowo: ,,Praca dzieli się na $7$ rozdziałów (\dots)''. Rozdział \ref{chapter:politechnika} dotyczy (\dots). Temat został rozwinięty~w~\ref{chapter:podrozdzial}.