\Chapter{Introduction}\label{chapter:introduction}
% Start with stories
Since ancient times humans loved stories.
Some stories describe historical events, others are works of fiction.
Some inspire and teach, others let one escape from hardships of day to day life.
All stories have one thing in common.
One thing differentiates them from other written forms.
That thing is the existence of a plot.
% Define plot
Plot is a collection of events, usually in chronological order.
A plot can be realistic, based in history or it can be entirely made up.
It can be said that it is the plot that makes the story.
It contains all the characters, their actions and thoughts as well as events within the story.

% How were stories told
Before people invented the first writing system stories were being told mouth-to-mouth.
Oral traditions continued even after the era of the written word.
Most people have heard the tales of Odysseus and his epic journey.
It is commonly attributed to the ancient poet Homer.
His work inspired both ancient and contemporary historians\cite{marincola2007odysseus}.
% Books and language
After writing got popular among common folk, the spread of storytelling in its written form took off.
In the current day, anyone can write and share the fruits of one's creative imagination with the rest of the world.
Even languages are becoming less of a barrier as not only a large portion of the population speaks English but translation technology and globalization make intercultural and international communication easier than ever\cite{coulmas1987speak}.
But writing itself is not the only way to tell a story.

% Theater
Going back to ancient Greeks, one can find the wealth of plays written to be performed on stage in the first theaters or, more precisely, amfitheaters.
Actors on stage wore masks that obscured their real faces and instead displayed a chosen emotion.
This allowed the people watching from the far seats to still understand the scene.
The aim of these performances was called katharsis\cite{hart2010art}.
Soon, the theater as is known today became born and people have been writing plays for a multidute of reasons.
% Painting
% Radio
% Cinema
% Television
% Diversity of representation in media
Books, plays, television and radio are all passive forms of media where content can be consumed by everyone in the same way.
People are diverse in their appearance, ideologies, beliefs and behaviours.
There are many groups of people with different cultures, history and traditions.
One of the main problems with what is commonly known as mainstream media (that is media that is most popular) is that it caters to the majority of its consumers.
In the old days when writing was expensive the common people could not access any books and much less find their representation in them.
Even among the people beloging to the same culture, there are many tastes and opinions.
% Interactive storytelling
People like to engage in the stories they consume and the easiest way to achieve deeper engagement is through the ability to influence the direction of the story itself.
Authors such as Chris Crawford\cite{crawford2013interactive} have written about interactive storytelling in the context of current day computer games but the concept itself is really much older than that.
In 1930 Doris Webster and Mary Alden Hopkins wrote a book titled "Consider the Consequences!".
According to the authors the book had multiple endings which depended on the "taste of the individual reader" \cite{webster1930consequences}.
These gamebooks as they have been named (these days also referred to as "Choose Your Own Adventure" type stories\cite{kraft1981cyoa}) have evolved alongside tabletop roleplaying games.
One notable example of the genre is "Dungeons and Dragons", a game of which many people have heard even if they haven't played it themselves \cite{gygax1974dungeons}.
In the current day, the genre is very diverse and consists of a multitude of games, some commercially available, others entirely fan-made.

%
This shows that interactive storytelling is a unique way of engaging and captivating audiences by allowing them to actively participate in the story.
In this form of storytelling, the audience is not just passive listeners, but active participants who are given the opportunity to make choices and influence the direction of the narrative.
Interactive storytelling increases audience engagement and immersion in the story. When the audience becomes part of the narrative, they develop a sense of ownership and become more invested in the story.
This level of immersion can lead to a more meaningful and memorable experience.
Interactive storytelling allows the audience to step into the shoes of the characters in the story and experience their emotions and decisions.
This can foster empathy and a deeper understanding of different perspectives and experiences.
By participating in the story, the audience can better appreciate the complexity of the characters' motives and actions.
Interactive storytelling can be personalized to the audience's interests and preferences.
By offering choices and different paths, the audience can tailor the story to their liking.
This level of personalization can increase engagement and create a more satisfying experience.
Interactive storytelling can encourage creativity and imagination.
By allowing the audience to make choices and shape the story, they are given a sense of agency and control.
This can inspire them to explore different paths and possibilities, leading to a more creative and imaginative experience.
In conclusion, interactive storytelling has several benefits, including enhancing engagement and immersion, fostering empathy, enhancing learning, offering personalization, increasing retention and recall, and enhancing creativity.
Interactive storytelling can be a powerful tool for creating memorable and impactful experiences for audiences of all ages and backgrounds.
%

Especially in the case of aforementioned tabletop roleplaying games, the players have nearly infinite freedom of expression and ability to influence the world limited only by their creativity.
This works only because the world exists in the minds of the players and all interactions can be simulated in the head of the player acting as the game master (a person responsbile for coordinating the players actions and telling the interactive story).
The advent of computer games brought unparalleled ability for artists to allow others to become immersed in their fictional worlds.
Current advances in computer graphics make it possible to display life-like images of the game world.
No longer does a player need to rely solely on their imagination to be able to immersive themselves in the story.
While this comes with many benefits, the major downside is that the freedom to interact with the game is limited by what the developers of the game have forseen.
Linear stories akin to movies are simple to develop while large scale open-world games are difficult.
Because of that, many games offer very shallow mechanics of interaction.
The choices that are offered to the player are often of no consequence and only exist to create an illusion that will deepen the player's engagement and immersion.

One particular example of this is the ability to break the law of the simulated world.
A player may choose to steal an apple from a market stall and usually a bunch of city guards will take pursuit.
This situation illustrates the consequence of the action undertaken by the player.
This illusion breaks however when the player was never seen stealing and then travels half the continent and tries to sell it just to be told that the vendor will not accept stolen goods.
Simplifications made in the simulation of game worlds are necessary and dictated by hardware limitations.
The same holds for choices made by the player during the game.
If a major political figure is taken captive the player may make a choice to free them and earn their trust or allow the bandits to keep them incarcerated.
Such a choice should affect the rest of the world and its impact should be growing stronger as time passes to eventaully die out.
In reality howevever, most games do not feature such a complex interaction system.
There are exceptions to this such as "The Witcher 3" which allows the player to influence the state of the world by making decisions in certain points of the game \cite{vickery2018directing}.
Even then, the narrative in such games always follows one of the predetermined branches.
% Interactive media
% lepsze utożsamianie się z bohaterem, możliwość kszatłtowania historii, w momencie gdy mamy wpływ, stajemy się bohaterem
% reprezentacja grup społecznych
% dopamina - osiągnięcia bohaterów stają się naszymi
% potrzeba tworzenia sztuki i opowiadania historii, na piramidzie potrzeb jest potrzeba samorealizacji, w tym tworzenia
% Games
% Information

People experience reality through the lens of their accumulated experience.
Each event in the life of a person adds another perspective on any following event.
Our behaviours are dictated by the way we perceive reality.
That perception is a combination of external sensory information and internal state of mind.
Our emotions, values and beliefs can fundamentally change the interpretation of any given situation.
As we grow old we look back on the choices we made and judge them by the standards of the present.
This simple fact means that there is no objectivity in how humans perceive reality.
Some situations or statements are simply more common in interpretation that others and thus we call them facts.
Others we refer to as opinions and some can be called delusions.
In games, the player is often tasked with accomplishing a goal which puts the future of the world at stake.
The choices made by the player should feel impactful to the overall narrative or the player will feel disengaged from the story, breaking immersion.
Often, choices can be morally ambiguous and the player should face the consequences of their actions in one way or another.
In society crimes are punishable on the basis of proof.
Proof is knowledge backed up by evidence, ergo a type of transferrable information.
An eyewitness of a crime can be untrustworthy and thus a crime might not be punished.
In games this mechanism is often simplified and a crime is detected and propagated across all the land if the player is briefly witnessed even by a single person.
The same goes for other player choices and game events.
People start referencing in-game events in dialogues instantly after they happen as if seemingly everyone became aware of them instantenously.
This can break immersion and player expectations.

\section{Problem Statement}
Propagation of knowledge or in general information is often too simplified in games.
Player actions should generate information that can be perceived, interpreted and propagated naturally by the other agents in the simulated game world.
Interpretation of received (objective) information produces (subjective) knowledge.
The distinction allows for effects such as misinterpretation and thus propagation of false information in the system.
This can lead to interesting mechanics and interactions that simulate reality more closly, increasing player engagement and immersion.
The main topic of research of this thesis is analysis of the three aspects of information in a game world, namely:
% Add also representation of knowledge
\begin{description}
    \item[perception] how, what and when information is received
    \item[interpretation] how information becomes knowledge
    \item[propagation] how knowledge becomes information and propagates to other people
\end{description}

The analysis as well as simulations and other research takes into account the overall end goal of the thesis being implementation in games where computing power is limited and real-time performance is paramount.
In a world with access to information in real-time across the globe via the use of internet, all models will be inherently different in comparison to a world without any such technology.
This thesis takes into account medieval style worlds with fantasy elements such as magic or mythical creatures but no advanced communication techniques.

\section{Thesis Objectives}
To properly understand the three aspects of information previously mentioned, the distinction between information and knowledge needs to be made.
Information itself needs to be defined as well.
To model the behaviour of different models, a cellular automata approach is chosen.

\section{Thesis Outline}
Zarysuj strukturę swojej pracy dyplomowej. Ogólnie przedstawienie pracy. Przykładowo: ,,Praca dzieli się na $7$ rozdziałów (\dots)''. Rozdział \ref{chapter:politechnika} dotyczy (\dots). Temat został rozwinięty~w~\ref{chapter:podrozdzial}.