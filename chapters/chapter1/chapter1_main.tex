\Chapter{Information versus Knowledge}

% What is information?
The most important distinction when building information propagation models is the one that differentiates knowledge from information.
The latter can be propagated while the former cannot.
This is due to the fact that information represents a statement of fact, irrespective of whether it is true or not.
A claim that people with axes are walking down the road towards the forest can be an information.
It should be considered to be fundamentally literal and represent nothing more than the messaging mechanism used to convey it.

% Knowledge
A person has a mind capable of linking events, seeing connections between facts, identifying patterns and inferring new knowledge from received information.
Anyone can easily see that such a group of aforementioned people can represent lumberjacks aiming to chop down the trees.
Note that neither the word lumberjack or the act of chopping down any tree was mentioned before.
This shows the distinction between what is pure information and what can be considered knowledge.
After all, how does one come to the conclusion that axes and trees must mean lumberjacks?
One needs to be aware of the concept of an axe and its function as well as the existence of the trade of forestry to make that connection.

% Propagation
After realizing the meaning of people with axes going towards a forest, one can decide to tell others of this news by formulating a new statement of fact: "Lumberjacks are going to the forest".
A more natural example would be a person telling a friend that someone told them about a group of lumberjacks heading to a forest.
This is an example of information propagation.
It can be illustrated more clearly by assuming the person who heard about the people heading towards the forest had recently seen a pack of wolves in the area.
They might get alarmed and wish to warn the unsuspecting workers.
The information "there are wolves in the forest" has the characteristic of urgency as well as importance.
It is urgent because telling it after the workers get to the forest misses the whole point of warning them.
It is important because it influences their safety.
These two factors lay the foundation for best possible propagation effect.
One would immedietely embark on a trip to warn the lumberjacks of the dangers lurking in the forest and probably mention the situation to anyone they pass by.
This particular situation also has the characteristic of being targeted towards a specific group.

% TODO: Find characteristics of information in literature
% Information characteristics
Information can be urgent, important and have a target group of receiving people.
Urgency determines the temporal aspect of propagation.
Importance influences how much effort one is willing to spend delivering the information.
A target audience limits the list of potential receivers.
An information can be urgent and not important, that would simply mean that it will expire after some limited time but if the effort to deliver it on time is too great, it is possible that it won't be done.
It can be important but not urgent just as well.
In this case it means that it must be delivered but there are no consequences to doing so later rather than sooner.
It is worth noting however that importance also influences the temporal element of information propagation as more important messages are going to be propagated faster.
A target audience can be homogeneus (a group of lumberjacks) or have a variety of different people.
A person doing the propagation can have a list of intended recipents of a message they are carrying.
There can also be people who are subscribers of selected kinds of information.
A mother would like to know if anything happens to her child and as such one can say she subscribes information regarding the health of her children.
In real life there are many more aspects and characteristcs that are deeper and more nuanced that what was described.
No artifical model can represent accurately the intricacies of the human mind, much less a group of minds.

% Complexity of information systems
An information system is a set of actors possessing knowledge and wishing to spread it in the form of information.
Because the complexity of real information systems is too great, simplification needs to be made in order to focus on the important aspects of such a system.
We can identify the key elements of any information system:
\begin{itemize}
    \item Actor - a person capable of processing information, holds knowledge and can produce new information
    \item Information - statement originating from any information source, ie. an actor
    \item Information source - any source of information, can be an actor or an inanimate source such as a recording or text
    \item Information pathway - any link between actors through which information can travel, can affect the information causing distortion or impact the propagation
\end{itemize}