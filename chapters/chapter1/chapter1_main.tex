\Chapter{Information versus Knowledge}

% What is information?
The most important distinction when building information propagation models is the one that differentiates knowledge from information.
The latter can be propagated while the former cannot.
This is due to the fact that information represents a statement of fact, irrespective of whether it is true or not.
A claim that people with axes are walking down the road towards the forest can be an information.
It should be considered to be fundamentally literal and represent nothing more than the messaging mechanism used to convey it.

% Knowledge
A person has a mind capable of linking events, seeing connections between facts, identifying patterns and inferring new knowledge from received information.
Anyone can easily see that such a group of aforementioned people can represent lumberjacks aiming to chop down the trees.
Note that neither the word lumberjack or the act of chopping down any tree was mentioned before.
This shows the distinction between what is pure information and what can be considered knowledge.
After all, how does one come to the conclusion that axes and trees must mean lumberjacks?
One needs to be aware of the concept of an axe and its function as well as the existence of the trade of forestry to make that connection.

% Propagation
After realizing the meaning of people with axes going towards a forest, one can decide to tell others of this news by formulating a new statement of fact: "Lumberjacks are going to the forest".
A more natural example would be a person telling a friend that someone told them about a group of lumberjacks heading to a forest.
This is an example of information propagation.
It can be illustrated more clearly by assuming the person who heard about the people heading towards the forest had recently seen a pack of wolves in the area.
They might get alarmed and wish to warn the unsuspecting workers.
The information "there are wolves in the forest" has the characteristic of urgency as well as importance.
It is urgent because telling it after the workers get to the forest misses the whole point of warning them.
It is important because it influences their safety.
These two factors lay the foundation for best possible propagation effect.
One would immedietely embark on a trip to warn the lumberjacks of the dangers lurking in the forest and probably mention the situation to anyone they pass by.
This particular situation also has the characteristic of being targeted towards a specific group.

% TODO: Find characteristics of information in literature
% Information characteristics
Information can be urgent, important and have a target group of receiving people.
Urgency determines the temporal aspect of propagation.
Importance influences how much effort one is willing to spend delivering the information.
A target audience limits the list of potential receivers.
An information can be urgent and not important, that would simply mean that it will expire after some limited time but if the effort to deliver it on time is too great, it is possible that it won't be done.
It can be important but not urgent just as well.
In this case it means that it must be delivered but there are no consequences to doing so later rather than sooner.
It is worth noting however that importance also influences the temporal element of information propagation as more important messages are going to be propagated faster.
A target audience can be homogeneus (a group of lumberjacks) or have a variety of different people.
A person doing the propagation can have a list of intended recipents of a message they are carrying.
There can also be people who are subscribers of selected kinds of information.
A mother would like to know if anything happens to her child and as such one can say she subscribes information regarding the health of her children.
In real life there are many more aspects and characteristcs that are deeper and more nuanced that what was described.
No artifical model can represent accurately the intricacies of the human mind, much less a group of minds.

% How information and knowledge interact
Information can not exist without someone to perceive it and knowledge can't exist without information.
Whenever there is some entity that is able to receive information and understand it, it can produce knowledge.
Information nececessitates a receiver to exist somewhere in the system that can produce knowledge.
This means that all three must always be present.
Collections of actors able to receive information and process it into knowledge as well as create new information and share it with other actors will be henceforth reffered to as information systems.

% Complexity of information systems
An information system is a set of actors possessing knowledge and wishing to spread it in the form of information.
Because the complexity of real information systems is too great, simplification needs to be made in order to focus on the important aspects of such a system.
We can identify the key elements of any information system:
\begin{itemize}
    \item Actor - a person capable of processing information, holds knowledge and can produce new information
    \item Information - statement originating from any information source, ie. an actor
    \item Information source - any source of information, can be an actor or an inanimate source such as a recording or text
    \item Information pathway - any link between actors through which information can travel, can affect the information causing distortion or impact the propagation
\end{itemize}
Actors may be offered information or they might produce information themselves.
This must always happen through an information pathway.
Inanimate sources of information, ie a written book, may exist within the same system and are considered to be always producing some (usually constant) information.
An actor may react differently when offered different kinds of information or even the same one repeatedly.
To illustrate that one can imagine being interested in buying a new pair of shoes if seeing the advertisment of them for the first time and getting more annoyed each consecutive time the same advertisment is played.
% TODO: Cite the work modelling marketing strategy effectiveness
This behaviour is usually the target of marketing strategy analysis as simply increasing the intensiveness of advertisment broadcast will not increase the demand for the shown product.
Aside from being able to receive information, an actor may decide to share it with other actors.
They might decide to choose a target group or broadcast it to everyone they can reach.
Some actors may seek certain types of information and query for it anytime they can.
In order to do so, actors must possess some knowledge.

% Complexity of knowledge
Knowledge is a set of statements about the world that are consistent (irrespective of truthfulness).
One can claim that the planet is flat while another person might disagree and it is valid to say that both possess knowledge about the shape of the world.
A person can then infer new facts on the basis of what they already know and for instance make a claim that travelling past the edge of the Earth will cause the daring explorer to fall to their demise.
The second person will then receive such information and basing on their opposing set of known statements will reject it and possibly reduce the opinion of the first person, being less likely to believe anything they might then tell them.
It is however impossible to model the interactions between information necessary to infer new information the same way that happens in the human mind, at the very least not in any feasible way.
The complexity of interconnectedness of knowledge and information alike is too great to be represented fully.
One can however make assumptions and put constraints on the type of statements allowed and the knowledge that can be gained and thus create a simplistic representation of an information system.

% Scope - logical vs spatial
There are two main types of information systems that can be used to model information propagation:
% TODO: Add some citation to "podział przestrzeni na abstrakcyjną i fizyczną"
\begin{itemize}
    \item spatial - they model information spread across space, usually partitioned into homogeneus segments or tiles
    \item logical - representing connections between actors without taking into account physical relationships between them
\end{itemize}
The first kind of information system, spatial information systems (SIS) can be used to model large scale events happening across the world.
The most important assumption in these kind of systems is the homogenuity of any given unit of space as defined by the chosen partitioning method.
They assume that information has some kind of physical propagation characteristic that allows it to travel across space.
In these systems an actor needs not to be represented individually but rather can be treated as part of a homogeneus group.
This model might be used to represent how news regarding the death of a ruler might spread from the captial to the most remote villages.
The second type of information systems, logical information systerms (LIS) work on the basis of individual actors and interactions between them.
They are ill suited for large scale modelling as the number of actors to be considered would grow considerably.
Of course, a spatial system can be represented as a logical system if some of the more important regions of the spatial model are chosen and represented as actors in the logical model.

% What does it mean for information to be considered as propagated?
The process of propagating information from one actor to another can be considered in the scope of physical information receival as well as logical aspect of interpretation.
The act of knowledge assimilation is inherently present in the process of actor receiving information from any external source.
% Is it sufficient to transfer information from source to target?
Information can reach its target in a plethora of ways.
In many models
% What happens after information reaches the target?
% What are the sources of information?