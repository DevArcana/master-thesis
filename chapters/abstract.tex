\section*{Streszczenie}

Niniejsza praca magisterska bada zastosowanie systemu wieloagentowego z architekturą kognitywną do symulacji propagacji informacji, w szczególności w ramach epidemiologicznego modelu przedziałowego. Tradycyjne zachowanie oparte na skryptach wymaga obszernego ręcznego skryptowania dla każdej interakcji, co czyni je uciążliwym i ograniczającym w radzeniu sobie z różnymi sytuacjami. Z kolei zachowanie emergentne pozwala na bardziej dynamiczne i świadome kontekstu reakcje. Badania te wykorzystują system wieloagentowy do symulacji poszczególnych agentów z ich własnymi zachowaniami, ułatwionymi przez architekturę kognitywną. Celem niniejszej rozprawy jest wdrożenie i opisanie systemu wieloagentowego wykorzystującego architekturę kognitywną oraz porównanie jego skuteczności z modelem epidemiologicznym. Praca składa się z czterech rozdziałów, rozpoczynających się od przeglądu powiązanych prac. W rozdziale 2 omówiono systemy wieloagentowe i różne architektury kognitywne, co doprowadziło do wyboru i wdrożenia konkretnej architektury w rozdziale 3. W tym rozdziale przedstawiono również wyniki eksperymentów. Wreszcie, rozdział 4 kończy rozprawę i zawiera sugestie dotyczące przyszłych badań.

\section*{Abstract}

This master thesis explores the application of a multi-agent system with a cognitive architecture for simulating the propagation of information, specifically within an epidemiological compartmental model. Traditional scripted behavior requires extensive manual scripting for each interaction, making it cumbersome and limiting in handling diverse situations. In contrast, emergent behavior allows for more dynamic and context-aware reactions. This research leverages a multi-agent system to simulate individual agents with their own behaviors, facilitated by a cognitive architecture. The objectives of this thesis are to implement and describe a multi-agent system using a cognitive architecture and compare its effectiveness with an epidemiological model. The thesis consists of four chapters, starting with a review of related works. Chapter 2 discusses multi-agent systems and different cognitive architectures, leading to the selection and implementation of a specific architecture in Chapter 3. Experimental results are presented in this chapter as well. Finally, Chapter 4 concludes the thesis and provides suggestions for future research.