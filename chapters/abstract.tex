\section*{Streszczenie}

Niniejsza praca magisterska bada zastosowanie systemu wieloagentowego z architekturą kognitywną do reprezentacji stanu fabularnego oraz symulacji propagacji wydarzeń fabularnych. Stan fabularny świata wraz z jego wydarzeniami jest reprezentowany przez rozproszony system agentów, z których każdy ma wiedzę o wybranych wydarzeniach fabularnych. Tradycyjne zachowanie oparte na skryptach wymaga obszernego ręcznego skryptowania dla każdej interakcji, co czyni je uciążliwym i ograniczającym w radzeniu sobie z różnymi sytuacjami. Z kolei zachowanie emergentne pozwala na bardziej dynamiczne i świadome kontekstu reakcje. Badania te wykorzystują system wieloagentowy do symulacji poszczególnych agentów z ich własnymi zachowaniami, rozszerzonymi przez zastosowanie architektury kognitywnej. Celem niniejszej pracy jest wdrożenie i opisanie systemu wieloagentowego wykorzystującego architekturę kognitywną oraz porównanie jego skuteczności z modelem epidemiologicznym wykorzystanym jako model propagacji informacji.

\section*{Abstract}

This master thesis explores the application of a multi-agent system with a cognitive architecture for representation of plot state and simulation of propagation of plot events. The plot state is represented using a distributed system of agents in which each agent has possession of chosen plot events. Traditional scripted behavior requires extensive manual scripting for each interaction, making it cumbersome and limiting in handling diverse situations. In contrast, emergent behavior allows for more dynamic and context-aware reactions. This research leverages a multi-agent system to simulate individual agents with their own behaviors, facilitated by a cognitive architecture. The objectives of this thesis are to implement and describe a multi-agent system using a cognitive architecture and compare its effectiveness with an epidemiological model used as an information propagation model.