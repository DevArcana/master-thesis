\Chapter{Plot event propagation}

\section{Temporal structures in games}

In order to simulate with any degree of precision the evolution of plot state in a game, one must take into account the inherent time structures used in said game.
These temporal structures are inherently tied to the game rules which limit the scope of a valid game space and thus influence the rate and shape of plot evolution.
Lindley describes moves in games as "an abstraction over player action, mapping action to a specific significance within the rule set and independent of local, personal and idiosyncratic variations in performance" \cite{lindley2005story}.
They illustrate this by an example of the game of chess where a player may make a move when it is their turn and the space of available moves is limited to a very small subset of all valid chess moves.
There are two major types of time structures in games:

\begin{itemize}
    \item discrete - turns, moves, matches, rounds, etc
    \item continuous - simulation of the passage of time, can use simulation steps with fixed unit of time in-between or dynamic
\end{itemize}

The first kind of temporal structures is easy to reason about and model because it is computationally less intensive than simulating the world and in turn the evolution of plot state as often as possible (or at the very least very frequently to keep the illusion of continuity).
The second kind is much often encountered in contemporary video games which often include physics, weather and daytime simulation as well.
For the sake of simulating propagation of plot events, one can choose to artifically divide the continuous time structure of the game into discrete simulation steps with fixed timestep or rely on the event based model.
An example of such event would be two characters meeting which would trigger the simulation of plot state evolution between them.
The second approach is hard to measure and model without introducing a framework of simulation capable of simulating arbitrary routines of actors within the game world.
For this reason and because the first approach can be adapted to the second scenario as well, the approach described in the subsequent sections deals only with discrete time structures.

\section{Propagating plot events in logical models}

WIP

\section{Propagating plot events in spatial models}

WIP

\section{Crossing the boundary between spatial and logical model}

WIP