\Chapter{Representation of information}

Modelling propagation of information itself is possible without looking at how it is represented.
The same is not true for modelling knowledge and its interaction with the former.
The ability to express any concept relies on the application of language.
It is precisely that which enables and also limits everything that is able to be conveyed using it.
Natural language is very complicated and full of inconsistencies.
Human brain handles that very well and in fact often is the source of said inconsistencies.
In order to create a model useful in terms of game development, it needs to have a clearly defined way to represent information.
One such option is the application of relational algebra.
To create a language one needs to think about what concepts it should allow expression of.
For example, in the English language a very simple sentence can be assembled: "The apple is red".
This sentences implies several things about the state of the described world.
First and foremost it implies the existence of a specific apple.
Secondly, it implies that apples have the attribute of color and that this particular one is red.
Due to its simple nature, both semantically and syntatically, one could write the following mathematical representation of this sentence:

$$
    \left( apple, is, red \right)
$$

An ordered set of words that in the right context is understood to mean that there exists an apple and that said apple is red.
A reasonable assumption is that by having a sufficiently sophisticated representation, it should be possible to tell whether a given pair of such statements are conflicting.

$$
    \left( apple, is, red \right) \neq \left( apple, is, green \right)
$$

To a human such an equation will seem correct at first glance.
However, as previously mentioned, the word "red" carries an impliciation of the described object possessing the attribute of color.

$$
    \left( apple, is, red \right) \neq \left( apple, is, round \right)
$$

The above statement is false as it tries to compare two distinct attributes, namely shape and color.
In order to make these kinds of statements more explicit, a fourth element is added.

$$
    \left( apple, is, color, red \right)
$$

This way it is possible to represent any arbitrary statement in the form of: "Object $X$ has attribute $Y$ of value $Z$" in the form of:

$$
    \left( X, is, Y, Z \right)
$$

In terms of relational algebra one can think of the word "is" as an operator denoting a binary relation between some object and the value of some attribute.
Because a language should have more than a single type of expressable relation, the mathematical representation should be more generic to accomodate all kinds of binary relations:

$$
    \left( operator, first, second \right)
$$

In the case of the $is$ operator:

$$
    \left( is, object, \left\{ attribute, value \right\} \right)
$$

Following the previous example:

$$
    \left( is, apple, \left\{ color, red \right\} \right)
$$

Then, one can define a function determining whether two statements $S_a$ and $S_b$ are conflicting:

$$
    S_a = \left( is, x_a, \left\{ y_a, z_a \right\} \right)
$$

$$
    S_b = \left( is, x_b, \left\{ y_b, z_b \right\} \right)
$$

$$
    conflict \left( S_a, S_b \right) \iff x_a = x_b \land y_a = y_b \land z_a \neq z_b
$$

% Identity
One important aspect of the natural language is ambiguity of subject's identity.